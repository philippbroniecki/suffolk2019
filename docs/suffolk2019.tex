\documentclass[]{article}
\usepackage{lmodern}
\usepackage{amssymb,amsmath}
\usepackage{ifxetex,ifluatex}
\usepackage{fixltx2e} % provides \textsubscript
\ifnum 0\ifxetex 1\fi\ifluatex 1\fi=0 % if pdftex
  \usepackage[T1]{fontenc}
  \usepackage[utf8]{inputenc}
\else % if luatex or xelatex
  \ifxetex
    \usepackage{mathspec}
  \else
    \usepackage{fontspec}
  \fi
  \defaultfontfeatures{Ligatures=TeX,Scale=MatchLowercase}
\fi
% use upquote if available, for straight quotes in verbatim environments
\IfFileExists{upquote.sty}{\usepackage{upquote}}{}
% use microtype if available
\IfFileExists{microtype.sty}{%
\usepackage{microtype}
\UseMicrotypeSet[protrusion]{basicmath} % disable protrusion for tt fonts
}{}
\usepackage[margin=1in]{geometry}
\usepackage{hyperref}
\hypersetup{unicode=true,
            pdftitle={Suffolk 2019},
            pdfborder={0 0 0},
            breaklinks=true}
\urlstyle{same}  % don't use monospace font for urls
\usepackage{color}
\usepackage{fancyvrb}
\newcommand{\VerbBar}{|}
\newcommand{\VERB}{\Verb[commandchars=\\\{\}]}
\DefineVerbatimEnvironment{Highlighting}{Verbatim}{commandchars=\\\{\}}
% Add ',fontsize=\small' for more characters per line
\usepackage{framed}
\definecolor{shadecolor}{RGB}{248,248,248}
\newenvironment{Shaded}{\begin{snugshade}}{\end{snugshade}}
\newcommand{\AlertTok}[1]{\textcolor[rgb]{0.94,0.16,0.16}{#1}}
\newcommand{\AnnotationTok}[1]{\textcolor[rgb]{0.56,0.35,0.01}{\textbf{\textit{#1}}}}
\newcommand{\AttributeTok}[1]{\textcolor[rgb]{0.77,0.63,0.00}{#1}}
\newcommand{\BaseNTok}[1]{\textcolor[rgb]{0.00,0.00,0.81}{#1}}
\newcommand{\BuiltInTok}[1]{#1}
\newcommand{\CharTok}[1]{\textcolor[rgb]{0.31,0.60,0.02}{#1}}
\newcommand{\CommentTok}[1]{\textcolor[rgb]{0.56,0.35,0.01}{\textit{#1}}}
\newcommand{\CommentVarTok}[1]{\textcolor[rgb]{0.56,0.35,0.01}{\textbf{\textit{#1}}}}
\newcommand{\ConstantTok}[1]{\textcolor[rgb]{0.00,0.00,0.00}{#1}}
\newcommand{\ControlFlowTok}[1]{\textcolor[rgb]{0.13,0.29,0.53}{\textbf{#1}}}
\newcommand{\DataTypeTok}[1]{\textcolor[rgb]{0.13,0.29,0.53}{#1}}
\newcommand{\DecValTok}[1]{\textcolor[rgb]{0.00,0.00,0.81}{#1}}
\newcommand{\DocumentationTok}[1]{\textcolor[rgb]{0.56,0.35,0.01}{\textbf{\textit{#1}}}}
\newcommand{\ErrorTok}[1]{\textcolor[rgb]{0.64,0.00,0.00}{\textbf{#1}}}
\newcommand{\ExtensionTok}[1]{#1}
\newcommand{\FloatTok}[1]{\textcolor[rgb]{0.00,0.00,0.81}{#1}}
\newcommand{\FunctionTok}[1]{\textcolor[rgb]{0.00,0.00,0.00}{#1}}
\newcommand{\ImportTok}[1]{#1}
\newcommand{\InformationTok}[1]{\textcolor[rgb]{0.56,0.35,0.01}{\textbf{\textit{#1}}}}
\newcommand{\KeywordTok}[1]{\textcolor[rgb]{0.13,0.29,0.53}{\textbf{#1}}}
\newcommand{\NormalTok}[1]{#1}
\newcommand{\OperatorTok}[1]{\textcolor[rgb]{0.81,0.36,0.00}{\textbf{#1}}}
\newcommand{\OtherTok}[1]{\textcolor[rgb]{0.56,0.35,0.01}{#1}}
\newcommand{\PreprocessorTok}[1]{\textcolor[rgb]{0.56,0.35,0.01}{\textit{#1}}}
\newcommand{\RegionMarkerTok}[1]{#1}
\newcommand{\SpecialCharTok}[1]{\textcolor[rgb]{0.00,0.00,0.00}{#1}}
\newcommand{\SpecialStringTok}[1]{\textcolor[rgb]{0.31,0.60,0.02}{#1}}
\newcommand{\StringTok}[1]{\textcolor[rgb]{0.31,0.60,0.02}{#1}}
\newcommand{\VariableTok}[1]{\textcolor[rgb]{0.00,0.00,0.00}{#1}}
\newcommand{\VerbatimStringTok}[1]{\textcolor[rgb]{0.31,0.60,0.02}{#1}}
\newcommand{\WarningTok}[1]{\textcolor[rgb]{0.56,0.35,0.01}{\textbf{\textit{#1}}}}
\usepackage{longtable,booktabs}
\usepackage{graphicx,grffile}
\makeatletter
\def\maxwidth{\ifdim\Gin@nat@width>\linewidth\linewidth\else\Gin@nat@width\fi}
\def\maxheight{\ifdim\Gin@nat@height>\textheight\textheight\else\Gin@nat@height\fi}
\makeatother
% Scale images if necessary, so that they will not overflow the page
% margins by default, and it is still possible to overwrite the defaults
% using explicit options in \includegraphics[width, height, ...]{}
\setkeys{Gin}{width=\maxwidth,height=\maxheight,keepaspectratio}
\IfFileExists{parskip.sty}{%
\usepackage{parskip}
}{% else
\setlength{\parindent}{0pt}
\setlength{\parskip}{6pt plus 2pt minus 1pt}
}
\setlength{\emergencystretch}{3em}  % prevent overfull lines
\providecommand{\tightlist}{%
  \setlength{\itemsep}{0pt}\setlength{\parskip}{0pt}}
\setcounter{secnumdepth}{5}
% Redefines (sub)paragraphs to behave more like sections
\ifx\paragraph\undefined\else
\let\oldparagraph\paragraph
\renewcommand{\paragraph}[1]{\oldparagraph{#1}\mbox{}}
\fi
\ifx\subparagraph\undefined\else
\let\oldsubparagraph\subparagraph
\renewcommand{\subparagraph}[1]{\oldsubparagraph{#1}\mbox{}}
\fi

%%% Use protect on footnotes to avoid problems with footnotes in titles
\let\rmarkdownfootnote\footnote%
\def\footnote{\protect\rmarkdownfootnote}

%%% Change title format to be more compact
\usepackage{titling}

% Create subtitle command for use in maketitle
\providecommand{\subtitle}[1]{
  \posttitle{
    \begin{center}\large#1\end{center}
    }
}

\setlength{\droptitle}{-2em}

  \title{Suffolk 2019}
    \pretitle{\vspace{\droptitle}\centering\huge}
  \posttitle{\par}
    \author{}
    \preauthor{}\postauthor{}
    \date{}
    \predate{}\postdate{}
  

\begin{document}
\maketitle

{
\setcounter{tocdepth}{2}
\tableofcontents
}
\hypertarget{about-this-course}{%
\section*{About this course}\label{about-this-course}}
\addcontentsline{toc}{section}{About this course}

This course is an introduction R, RStudio and statistics. Our primary aims are to get comfortable working with R and to be able to prepare, manipulate, analyse and visualise data..

\begin{center}\rule{0.5\linewidth}{\linethickness}\end{center}

Agenda

\hypertarget{introduction-to-r-and-rstudio}{%
\section{Introduction to R and RStudio}\label{introduction-to-r-and-rstudio}}

\hypertarget{learning-objectives}{%
\subsection{Learning objectives}\label{learning-objectives}}

In this session, we will have a look at R and RStudio. We will interact with both and use the various components of RStudio.

\hypertarget{what-is-r}{%
\subsubsection{What is R?}\label{what-is-r}}

R is an environment for statistical computing and graphics. RStudio is an editor or integrated development environment (IDE) that makes working with R much more comfortable.

To install R and RStudio on your computer, download both from the following sources:

\begin{itemize}
\tightlist
\item
  Download R from \href{https://cran.r-project.org}{The Comprehensive R Archive Network (CRAN)}
\item
  Download RStudio from \href{https://www.rstudio.com}{RStudio.com}
\end{itemize}

Keep both R and RStudio up to date. That means go online and check for newer versions. In case there are new versions, download those and re-install.

\hypertarget{rstudio}{%
\subsubsection{RStudio}\label{rstudio}}

Let's get acquainted with R. When you start RStudio for the first time, you'll see three panes:

\includegraphics{./img/rstudio_default.png}

\hypertarget{console}{%
\subsubsection{Console}\label{console}}

The Console in RStudio is the simplest way to interact with R. You can type some code at the Console and when you press ENTER, R will run that code. Depending on what you type, you may see some output in the Console or if you make a mistake, you may get a warning or an error message.

Let's familiarize ourselves with the console by using R as a simple calculator:

\begin{Shaded}
\begin{Highlighting}[]
\DecValTok{2} \OperatorTok{+}\StringTok{ }\DecValTok{4}
\end{Highlighting}
\end{Shaded}

\begin{verbatim}
[1] 6
\end{verbatim}

Now that we know how to use the \texttt{+} sign for addition, let's try some other mathematical operations such as subtraction (\texttt{-}), multiplication (\texttt{*}), and division (\texttt{/}).

\begin{Shaded}
\begin{Highlighting}[]
\DecValTok{10} \OperatorTok{-}\StringTok{ }\DecValTok{4}
\end{Highlighting}
\end{Shaded}

\begin{verbatim}
[1] 6
\end{verbatim}

\begin{Shaded}
\begin{Highlighting}[]
\DecValTok{5} \OperatorTok{*}\StringTok{ }\DecValTok{3}
\end{Highlighting}
\end{Shaded}

\begin{verbatim}
[1] 15
\end{verbatim}

\begin{Shaded}
\begin{Highlighting}[]
\DecValTok{7} \OperatorTok{/}\StringTok{ }\DecValTok{2}
\end{Highlighting}
\end{Shaded}

\begin{verbatim}
[1] 3.5
\end{verbatim}

\begin{longtable}[]{@{}ll@{}}
\toprule
\endhead
\begin{minipage}[t]{0.69\columnwidth}\raggedright
You can use the cursor or arrow keys on your keyboard to edit your code at the console:- Use the UP and DOWN keys to re-run something without typing it again- Use the LEFT and RIGHT keys to edit\strut
\end{minipage} & \begin{minipage}[t]{0.25\columnwidth}\raggedright
\includegraphics{./img/rstudio_cursorkeys.png}\strut
\end{minipage}\tabularnewline
\bottomrule
\end{longtable}

Take a few minutes to play around at the console and try different things out. Don't worry if you make a mistake, you can't break anything easily!

\hypertarget{functions}{%
\subsubsection{Functions}\label{functions}}

Functions are a set of instructions that carry out a specific task. Functions often require some input and generate some output. For example, instead of using the \texttt{+} operator for addition, we can use the \texttt{sum} function to add two or more numbers.

\begin{Shaded}
\begin{Highlighting}[]
\KeywordTok{sum}\NormalTok{(}\DecValTok{1}\NormalTok{, }\DecValTok{4}\NormalTok{, }\DecValTok{10}\NormalTok{)}
\end{Highlighting}
\end{Shaded}

\begin{verbatim}
[1] 15
\end{verbatim}

In the example above, \texttt{1,\ 4,\ 10} are the inputs and 15 is the output. A function always requires the use of parenthesis or round brackets \texttt{()}. Inputs to the function are called \textbf{arguments} and go inside the brackets. The output of a function is displayed on the screen but we can also have the option of saving the result of the output. More on this later.

\hypertarget{getting-help}{%
\subsubsection{Getting Help}\label{getting-help}}

Another useful function in R is \texttt{help} which we can use to display online documentation. For example, if we wanted to know how to use the \texttt{sum} function, we could type \texttt{help(sum)} and look at the online documentation.

\begin{Shaded}
\begin{Highlighting}[]
\KeywordTok{help}\NormalTok{(sum)}
\end{Highlighting}
\end{Shaded}

The question mark \texttt{?} can also be used as a shortcut to access online help.

\begin{Shaded}
\begin{Highlighting}[]
\NormalTok{?sum}
\end{Highlighting}
\end{Shaded}

\includegraphics{./img/rstudio_help.png}

Use the toolbar button shown in the picture above to expand and display the help in a new window.

Help pages for functions in R follow a consistent layout generally include these sections:

\begin{longtable}[]{@{}ll@{}}
\toprule
\endhead
Description & A brief description of the function\tabularnewline
Usage & The complete syntax or grammar including all arguments (inputs)\tabularnewline
Arguments & Explanation of each argument\tabularnewline
Details & Any relevant details about the function and its arguments\tabularnewline
Value & The output value of the function\tabularnewline
Examples & Example of how to use the function\tabularnewline
\bottomrule
\end{longtable}

\hypertarget{the-assignment-operator}{%
\subsubsection{The Assignment Operator}\label{the-assignment-operator}}

Now we know how to provide inputs to a function using parenthesis or round brackets \texttt{()}, but what about the output of a function?

We use the assignment operator \textbf{\texttt{\textless{}-}} for creating or updating objects. If we wanted to save the result of adding \texttt{sum(1,\ 4,\ 10)}, we would do the following:

\begin{Shaded}
\begin{Highlighting}[]
\NormalTok{myresult <-}\StringTok{ }\KeywordTok{sum}\NormalTok{(}\DecValTok{1}\NormalTok{, }\DecValTok{4}\NormalTok{, }\DecValTok{10}\NormalTok{)}
\end{Highlighting}
\end{Shaded}

The line above creates a new object called \texttt{myresult} in our environment and saves the result of the \texttt{sum(1,\ 4,\ 10)} in it. To see what's in \texttt{myresult}, just type it at the console:

\begin{Shaded}
\begin{Highlighting}[]
\NormalTok{myresult}
\end{Highlighting}
\end{Shaded}

\begin{verbatim}
[1] 15
\end{verbatim}

Take a look at the \textbf{Environment} pane in RStudio and you'll see \texttt{myresult} there.

\includegraphics{./img/rstudio_env.png}

To delete all objects from the environment, you can use the \textbf{broom} button as shown in the picture above.

We called our object \texttt{myresult} but we can call it anything as long as we follow a few simple rules. Object names can contain upper or lower case letters (\texttt{A-Z}, \texttt{a-z}), numbers (\texttt{0-9}), underscores (\texttt{\_}) or a dot (\texttt{.}) but all object names must start with a letter. Choose names that are descriptive and easy to type.

\begin{longtable}[]{@{}ll@{}}
\toprule
Good Object Names & Bad Object Names\tabularnewline
\midrule
\endhead
result & a\tabularnewline
myresult & x1\tabularnewline
my.result & this.name.is.just.too.long\tabularnewline
my\_result &\tabularnewline
data1 &\tabularnewline
\bottomrule
\end{longtable}

\hypertarget{sequences}{%
\subsubsection{Sequences}\label{sequences}}

We often need to create sequences when manipulating data. For instance, you might want to perform an operation on the first 10 rows of a dataset so we need a way to select the range we're interested in.

There are two ways to create a sequence. Let's try to create a sequence of numbers from 1 to 10 using the two methods:

\begin{enumerate}
\def\labelenumi{\arabic{enumi}.}
\tightlist
\item
  Using the colon \texttt{:} operator. If you're familiar with spreadsheets then you might've already used \texttt{:} to select cells, for example \texttt{A1:A20}. In R, you can use the \texttt{:} to create a sequence in a similar fashion:
\end{enumerate}

\begin{Shaded}
\begin{Highlighting}[]
\DecValTok{1}\OperatorTok{:}\DecValTok{10}
\end{Highlighting}
\end{Shaded}

\begin{verbatim}
 [1]  1  2  3  4  5  6  7  8  9 10
\end{verbatim}

\begin{enumerate}
\def\labelenumi{\arabic{enumi}.}
\tightlist
\item
  Using the \texttt{seq} function we get the exact same result:
\end{enumerate}

\begin{Shaded}
\begin{Highlighting}[]
\KeywordTok{seq}\NormalTok{(}\DataTypeTok{from =} \DecValTok{1}\NormalTok{, }\DataTypeTok{to =} \DecValTok{10}\NormalTok{)}
\end{Highlighting}
\end{Shaded}

\begin{verbatim}
 [1]  1  2  3  4  5  6  7  8  9 10
\end{verbatim}

The \texttt{seq} function has a number of options which control how the sequence is generated. For example to create a sequence from 0 to 100 in increments of \texttt{5}, we can use the optional \texttt{by} argument. Notice how we wrote \texttt{by\ =\ 5} as the third argument. It is a common practice to specify the name of argument when the argument is optional. The arguments \texttt{from} and \texttt{to} are not optional, se we can write \texttt{seq(0,\ 100,\ by\ =\ 5)} instead of \texttt{seq(from\ =\ 0,\ to\ =\ 100,\ by\ =\ 5)}. Both, are valid ways of achieving the same outcome. You can code whichever way you like. We recommend to write code such that you make it easy for your future self and others to read and understand the code.

\begin{Shaded}
\begin{Highlighting}[]
\KeywordTok{seq}\NormalTok{(}\DataTypeTok{from =} \DecValTok{0}\NormalTok{, }\DataTypeTok{to =} \DecValTok{100}\NormalTok{, }\DataTypeTok{by =} \DecValTok{5}\NormalTok{)}
\end{Highlighting}
\end{Shaded}

\begin{verbatim}
 [1]   0   5  10  15  20  25  30  35  40  45  50  55  60  65  70  75  80
[18]  85  90  95 100
\end{verbatim}

Another common use of the \texttt{seq} function is to create a sequence of a specific length. Here, we create a sequence from 0 to 100 with length 9, i.e., the result is a vector with 9 elements.

\begin{Shaded}
\begin{Highlighting}[]
\KeywordTok{seq}\NormalTok{(}\DataTypeTok{from =} \DecValTok{0}\NormalTok{, }\DataTypeTok{to =} \DecValTok{100}\NormalTok{, }\DataTypeTok{length.out =}  \DecValTok{9}\NormalTok{)}
\end{Highlighting}
\end{Shaded}

\begin{verbatim}
[1]   0.0  12.5  25.0  37.5  50.0  62.5  75.0  87.5 100.0
\end{verbatim}

Now it's your turn:

\begin{itemize}
\tightlist
\item
  Create a sequence of \textbf{odd} numbers between 0 and 100 and save it in an object called \texttt{odd\_numbers}
\end{itemize}

\begin{Shaded}
\begin{Highlighting}[]
\NormalTok{odd_numbers <-}\StringTok{ }\KeywordTok{seq}\NormalTok{(}\DecValTok{1}\NormalTok{, }\DecValTok{100}\NormalTok{, }\DecValTok{2}\NormalTok{)}
\end{Highlighting}
\end{Shaded}

\begin{itemize}
\tightlist
\item
  Next, display \texttt{odd\_numbers} on the console to verify that you did it correctly
\end{itemize}

\begin{Shaded}
\begin{Highlighting}[]
\NormalTok{odd_numbers}
\end{Highlighting}
\end{Shaded}

\begin{verbatim}
 [1]  1  3  5  7  9 11 13 15 17 19 21 23 25 27 29 31 33 35 37 39 41 43 45
[24] 47 49 51 53 55 57 59 61 63 65 67 69 71 73 75 77 79 81 83 85 87 89 91
[47] 93 95 97 99
\end{verbatim}

\begin{itemize}
\item
  What do the numbers in square brackets \texttt{{[}\ {]}} mean? Look at the number of values displayed in each line to find out the answer.
\item
  Use the \texttt{length} function to find out how many values are in the object \texttt{odd\_numbers}.

  \begin{itemize}
  \tightlist
  \item
    HINT: Try \texttt{help(length)} and look at the examples section at the end of the help screen.
  \end{itemize}
\end{itemize}

\begin{Shaded}
\begin{Highlighting}[]
\KeywordTok{length}\NormalTok{(odd_numbers)}
\end{Highlighting}
\end{Shaded}

\begin{verbatim}
[1] 50
\end{verbatim}

\hypertarget{scripts}{%
\subsubsection{Scripts}\label{scripts}}

The Console is great for simple tasks but if you're working on a project you would mostly likely want to save your work in some sort of a document or a file. Scripts in R are just plain text files that contain R code. You can edit a script just like you would edit a file in any word processing or note-taking application.

Create a new script using the menu or the toolbar button as shown below.

\includegraphics{./img/rstudio_newfile.png}

Once you've created a script, it is generally a good idea to give it a meaningful name and save it immediately. For our first session save your script as \textbf{seminar1.R}

\begin{longtable}[]{@{}ll@{}}
\toprule
\endhead
\begin{minipage}[t]{0.52\columnwidth}\raggedright
Familiarize yourself with the script window in RStudio, and especially the two buttons labeled \textbf{Run} and \textbf{Source}\strut
\end{minipage} & \begin{minipage}[t]{0.42\columnwidth}\raggedright
\includegraphics{./img/rstudio_script.png}\strut
\end{minipage}\tabularnewline
\bottomrule
\end{longtable}

There are a few different ways to run your code from a script.

\begin{longtable}[]{@{}ll@{}}
\toprule
\endhead
\begin{minipage}[t]{0.23\columnwidth}\raggedright
One line at a time\strut
\end{minipage} & \begin{minipage}[t]{0.71\columnwidth}\raggedright
Place the cursor on the line you want to run and hit CTRL-ENTER or use the \textbf{Run} button\strut
\end{minipage}\tabularnewline
\begin{minipage}[t]{0.23\columnwidth}\raggedright
Multiple lines\strut
\end{minipage} & \begin{minipage}[t]{0.71\columnwidth}\raggedright
Select the lines you want to run and hit CTRL-ENTER or use the \textbf{Run} button\strut
\end{minipage}\tabularnewline
\begin{minipage}[t]{0.23\columnwidth}\raggedright
Entire script\strut
\end{minipage} & \begin{minipage}[t]{0.71\columnwidth}\raggedright
Use the \textbf{Source} button\strut
\end{minipage}\tabularnewline
\bottomrule
\end{longtable}

\hypertarget{central-tendency}{%
\subsubsection{Central Tendency}\label{central-tendency}}

The appropriate measure of central tendency depends on the level of measurement of the variable. To recap:

\begin{longtable}[]{@{}ll@{}}
\toprule
Level of measurement & Appropriate measure of central tendency\tabularnewline
\midrule
\endhead
Continuous & arithmetic mean (or average)\tabularnewline
Ordered & median (or the central observation)\tabularnewline
Nominal & mode (the most frequent value)\tabularnewline
\bottomrule
\end{longtable}

\hypertarget{mean}{%
\paragraph{Mean}\label{mean}}

We calculate the average grade on our eleven homework assignments in statistics 1. We create our vector of 11 (fake) grades first using the \texttt{c()} function, where \texttt{c} stands for collect or concatenate.

\begin{Shaded}
\begin{Highlighting}[]
\NormalTok{hw.grades <-}\StringTok{ }\KeywordTok{c}\NormalTok{(}\DecValTok{80}\NormalTok{, }\DecValTok{90}\NormalTok{, }\DecValTok{85}\NormalTok{, }\DecValTok{71}\NormalTok{, }\DecValTok{69}\NormalTok{, }\DecValTok{85}\NormalTok{, }\DecValTok{83}\NormalTok{, }\DecValTok{88}\NormalTok{, }\DecValTok{99}\NormalTok{, }\DecValTok{81}\NormalTok{, }\DecValTok{92}\NormalTok{)}
\end{Highlighting}
\end{Shaded}

We now take the sum of the grades.

\begin{Shaded}
\begin{Highlighting}[]
\NormalTok{sum.hw.grades <-}\StringTok{ }\KeywordTok{sum}\NormalTok{(hw.grades)}
\end{Highlighting}
\end{Shaded}

We also take the number of grades

\begin{Shaded}
\begin{Highlighting}[]
\NormalTok{number.hw.grades <-}\StringTok{ }\KeywordTok{length}\NormalTok{(hw.grades) }
\end{Highlighting}
\end{Shaded}

The mean is the sum of grades over the number of grades.

\begin{Shaded}
\begin{Highlighting}[]
\NormalTok{sum.hw.grades }\OperatorTok{/}\StringTok{ }\NormalTok{number.hw.grades}
\end{Highlighting}
\end{Shaded}

\begin{verbatim}
[1] 83.90909
\end{verbatim}

R provides us with an even easier way to do the same with a function called \href{http://bit.ly/R_mean}{\texttt{mean()}}.

\begin{Shaded}
\begin{Highlighting}[]
\KeywordTok{mean}\NormalTok{(hw.grades)}
\end{Highlighting}
\end{Shaded}

\begin{verbatim}
[1] 83.90909
\end{verbatim}

\hypertarget{median}{%
\paragraph{Median}\label{median}}

The median is the appropriate measure of central tendency for ordinal variables. Ordinal means that there is a rank ordering but not equally spaced intervals between values of the variable. Education is a common example. In education, more education is better. But the difference between primary school and secondary school is not the same as the difference between secondary school and an undergraduate degree.

Let's generate a fake example with 100 people. We use numbers to code different levels of education.

\begin{longtable}[]{@{}lll@{}}
\toprule
\endhead
Code & Meaning & Frequency in our data\tabularnewline
0 & no education & 1\tabularnewline
1 & primary school & 5\tabularnewline
2 & secondary school & 55\tabularnewline
3 & undergraduate degree & 20\tabularnewline
4 & postgraduate degree & 10\tabularnewline
5 & doctorate & 9\tabularnewline
\bottomrule
\end{longtable}

We introduce a new function to create a vector. The function \texttt{rep()}, replicates elements of a vector. Its arguments are the item \texttt{x} to be replicated and the number of \texttt{times} to replicate. Below, we create the variable education with the frequency of education level indicated above. Note that the arguments \texttt{x} and \texttt{times} do not have to be written out.

\begin{Shaded}
\begin{Highlighting}[]
\NormalTok{edu <-}\StringTok{ }\KeywordTok{c}\NormalTok{( }\KeywordTok{rep}\NormalTok{(}\DataTypeTok{x=}\DecValTok{0}\NormalTok{, }\DataTypeTok{times=}\DecValTok{1}\NormalTok{), }\KeywordTok{rep}\NormalTok{(}\DataTypeTok{x=}\DecValTok{1}\NormalTok{, }\DataTypeTok{times=}\DecValTok{5}\NormalTok{), }\KeywordTok{rep}\NormalTok{(}\DataTypeTok{x=}\DecValTok{2}\NormalTok{, }\DataTypeTok{times=}\DecValTok{55}\NormalTok{),}
          \KeywordTok{rep}\NormalTok{(}\DataTypeTok{x=}\DecValTok{3}\NormalTok{, }\DataTypeTok{times=}\DecValTok{20}\NormalTok{), }\KeywordTok{rep}\NormalTok{(}\DecValTok{4}\NormalTok{,}\DecValTok{10}\NormalTok{), }\KeywordTok{rep}\NormalTok{(}\DecValTok{5}\NormalTok{,}\DecValTok{9}\NormalTok{) )}
\end{Highlighting}
\end{Shaded}

The median level of education is the level where 50 percent of the observations have a lower or equal level of education and 50 percent have a higher or equal level of education. That means that the median splits the data in half.

We use the \href{http://bit.ly/R_median}{\texttt{median()}} function for finding the median.

\begin{Shaded}
\begin{Highlighting}[]
\KeywordTok{median}\NormalTok{(edu)}
\end{Highlighting}
\end{Shaded}

\begin{verbatim}
[1] 2
\end{verbatim}

The median level of education is secondary school.

\hypertarget{mode}{%
\paragraph{Mode}\label{mode}}

The mode is the appropriate measure of central tendency if the level of measurement is nominal. Nominal means that there is no ordering implicit in the values that a variable takes on. We create data from 1000 (fake) voters in the United Kingdom who each express their preference on remaining in or leaving the European Union. The options are leave or stay. Leaving is not greater than staying and vice versa (even though we all order the two options normatively).

\begin{longtable}[]{@{}lll@{}}
\toprule
\endhead
Code & Meaning & Frequency in our data\tabularnewline
0 & leave & 509\tabularnewline
1 & stay & 491\tabularnewline
\bottomrule
\end{longtable}

\begin{Shaded}
\begin{Highlighting}[]
\NormalTok{stay <-}\StringTok{ }\KeywordTok{c}\NormalTok{(}\KeywordTok{rep}\NormalTok{(}\DecValTok{0}\NormalTok{, }\DecValTok{509}\NormalTok{), }\KeywordTok{rep}\NormalTok{(}\DecValTok{1}\NormalTok{, }\DecValTok{491}\NormalTok{))}
\end{Highlighting}
\end{Shaded}

The mode is the most common value in the data. There is no mode function in R. The most straightforward way to determine the mode is to use the \href{http://bit.ly/R_table}{\texttt{table()}} function. It returns a frequency table. We can easily see the mode in the table. As your coding skills increase, you will see other ways of recovering the mode from a vector.

\begin{Shaded}
\begin{Highlighting}[]
\KeywordTok{table}\NormalTok{(stay)}
\end{Highlighting}
\end{Shaded}

\begin{verbatim}
stay
  0   1 
509 491 
\end{verbatim}

The mode is leaving the EU because the number of `leavers' (0) is greater than the number of `remainers' (1).

\hypertarget{dispersion}{%
\subsubsection{Dispersion}\label{dispersion}}

The appropriate measure of dispersion depends on the level of measurement of the variable we wish to describe.

\begin{longtable}[]{@{}ll@{}}
\toprule
Level of measurement & Appropriate measure of dispersion\tabularnewline
\midrule
\endhead
Continuous & variance and/or standard deviation\tabularnewline
Ordered & range or interquartile range\tabularnewline
Nominal & proportion in each category\tabularnewline
\bottomrule
\end{longtable}

\hypertarget{variance-and-standard-deviation}{%
\paragraph{Variance and standard deviation}\label{variance-and-standard-deviation}}

Both the variance and the standard deviation tell by how much an average realisation of a variable differs from the mean of that variable. Let's assume that our variable is income in the UK. Let's assume that its mean is 35 000 per year. We also assume that the average deviation from 35 000 is 5 000. If we ask 100 people in the UK at random about their income, we get 100 different answers. If we average the differences betweeen the 100 answers and 35 000, we would get 5 000. Suppose that the average income in France is also 35 000 per year but the average deviation is 10 000 instead. This would imply that income is more equally distributed in the UK than in France.

Dispersion is important to describe data as this example illustrates. Although, mean income in our hypothetical example is the same in France and the UK, the distribution is tighter in the UK. The figure below illustrates our example:

\includegraphics{suffolk2019_files/figure-latex/unnamed-chunk-27-1.pdf}

The variance gives us an idea about the variability of data. The formula for the variance in the population is \[ \frac{\sum_{i=1}^n(x_i - \mu_x)^2}{n}\]

The formula for the variance in a sample adjusts for sampling variability, i.e., uncertainty about how well our sample reflects the population by subtracting 1 in the denominator. Subtracting 1 will have next to no effect if n is large but the effect increases the smaller n.~The smaller n, the larger the sample variance. The intuition is, that in smaller samples, we are less certain that our sample reflects the population. We, therefore, adjust variability of the data upwards. The formula is

\[ \frac{\sum_{i=1}^n(x_i - \bar{x})^2}{n-1}\]

Notice the different notation for the mean in the two formulas. We write \(\mu_x\) for the mean of x in the population and \(\bar{x}\) for the mean of x in the sample. Notation is, however, unfortunately not always consistent.

Take a minute to think your way through the formula. There are 4 setps: (1), In the numerator, we subtract the mean of x from some realisation of x. (2), We square the deviations from the mean because we want positive numbers only. (3) We sum the squared deviations. (4) We divide the sum by \((n-1)\). Below we show this for the homework example. In the last row, we add a 5th step. We take the square root in order to return to the orginial units of the homework grades.

\begin{longtable}[]{@{}llll@{}}
\toprule
Obs & Var & Dev. from mean & Squared dev. from mean\tabularnewline
\midrule
\endhead
i & grade & \(x_i-\bar{x}\) & \((x_i-\bar{x})^2\)\tabularnewline
1 & 80 & -3.9090909 & 15.2809917\tabularnewline
2 & 90 & 6.0909091 & 37.0991736\tabularnewline
3 & 85 & 1.0909091 & 1.1900826\tabularnewline
4 & 71 & -12.9090909 & 166.6446281\tabularnewline
5 & 69 & -14.9090909 & 222.2809917\tabularnewline
6 & 85 & 1.0909091 & 1.1900826\tabularnewline
7 & 83 & -0.9090909 & 0.8264463\tabularnewline
8 & 88 & 4.0909091 & 16.7355372\tabularnewline
9 & 99 & 15.0909091 & 227.7355372\tabularnewline
10 & 81 & -2.9090909 & 8.4628099\tabularnewline
11 & 92 & 8.0909091 & 65.4628099\tabularnewline
\(\sum_{i=1}^n\) & & & 762.9090909\tabularnewline
\(\div n-1\) & & & 76.2909091\tabularnewline
\(\sqrt{}\) & & & 8.7344667\tabularnewline
\bottomrule
\end{longtable}

Our first grade (80) is below the mean (83.9090909). The sum is, thus, negative. Our second grade (90) is above the mean, so that the sum is positive. Both are deviations from the mean (think of them as distances). Our sum shall reflect the total sum of distances and distances must be positive. Hence, we square the distances from the mean. Having done this for all eleven observations, we sum the squared distances. Dividing by 10 (with the sample adjustment), gives us the average squared deviation. This is the variance. The units of the variance---squared deviations---are somewhat awkward. We return to this in a moment.

We take the variance in R by using the \href{http://bit.ly/R_var}{\texttt{var()}} function. By default \texttt{var()} takes the sample variance.

\begin{Shaded}
\begin{Highlighting}[]
\KeywordTok{var}\NormalTok{(hw.grades)}
\end{Highlighting}
\end{Shaded}

\begin{verbatim}
[1] 76.29091
\end{verbatim}

The average squared difference form our mean grade is 76.2909091. But what does that mean? We would like to get rid of the square in our units. That's what the standard deviation does. The standard deviation is the square root over the variance.

\[ \sqrt{\frac{\sum_{i=1}^n(x_i - \bar{x})^2}{n-1}}\]

We get the average deviation from our mean grade (83.9090909) with the \href{http://bit.ly/R_sd}{\texttt{sd()}} function.

\begin{Shaded}
\begin{Highlighting}[]
\KeywordTok{sd}\NormalTok{(hw.grades)}
\end{Highlighting}
\end{Shaded}

\begin{verbatim}
[1] 8.734467
\end{verbatim}

The standard deviation is much more intuitive than the variance because its units are the same as the units of the variable we are interested in. ``Why teach us about this awful variance then'', you ask. Mathematically, we have to compute the variance before getting the standard deviation. We recommend that you use the standard deviation to describe the variability of your continuous data.

Note: We used the sample variance and sample standard deviation formulas. If the eleven assignments represent the population, we would use the population variance formula. Whether the 11 cases represent a sample or the population depends on what we want to know. If we want learn about all students' assignments or future assignments, the 11 cases are a sample.

\hypertarget{range-and-interquartile-range}{%
\paragraph{Range and interquartile range}\label{range-and-interquartile-range}}

The proper measure of dispersion of an ordinal variable is the range or the interquartile range. The interquartile range is usually the preferred measure because the range is strongly affected by outlying cases.

Let's take the range first. We get back to our education example. In R, we use the \href{http://bit.ly/R_range}{\texttt{range()}} function to compute the range.

\begin{Shaded}
\begin{Highlighting}[]
\KeywordTok{range}\NormalTok{(edu)}
\end{Highlighting}
\end{Shaded}

\begin{verbatim}
[1] 0 5
\end{verbatim}

Our data ranges from no education all the way to those with a doctorate. However, no education is not a common value. Only one person in our sample did not have any education. The interquartile range is the range from the 25th to the 75th percentiles, i.e., it contains the central 50 percent of the distribution.

The 25th percentile is the value of education that 25 percent or fewer people have (when we order education from lowest to highest). We use the \href{http://bit.ly/R_quantile}{\texttt{quantile()}} function in R to get percentiles. The function takes two arguments: \texttt{x} is the data vector and \texttt{probs} is the percentile.

\begin{Shaded}
\begin{Highlighting}[]
\KeywordTok{quantile}\NormalTok{(edu, }\FloatTok{0.25}\NormalTok{) }\CommentTok{# 25th percentile}
\end{Highlighting}
\end{Shaded}

\begin{verbatim}
25% 
  2 
\end{verbatim}

\begin{Shaded}
\begin{Highlighting}[]
\KeywordTok{quantile}\NormalTok{(edu, }\FloatTok{0.75}\NormalTok{) }\CommentTok{# 75th percentile}
\end{Highlighting}
\end{Shaded}

\begin{verbatim}
75% 
  3 
\end{verbatim}

Therefore, the interquartile range is from 2, secondary school to 3, undergraduate degree.

\hypertarget{proportion-in-each-category}{%
\paragraph{Proportion in each category}\label{proportion-in-each-category}}

To describe the distribution of our nominal variable, support for remaining in the European Union, we use the proportions in each category.

Recall, that we looked at the frequency table to determine the mode:

\begin{Shaded}
\begin{Highlighting}[]
\KeywordTok{table}\NormalTok{(stay)}
\end{Highlighting}
\end{Shaded}

\begin{verbatim}
stay
  0   1 
509 491 
\end{verbatim}

To get the proportions in each category, we divide the values in the table, i.e., 509 and 491, by the sum of the table, i.e., 1000.

\begin{Shaded}
\begin{Highlighting}[]
\KeywordTok{table}\NormalTok{(stay) }\OperatorTok{/}\StringTok{ }\KeywordTok{sum}\NormalTok{(}\KeywordTok{table}\NormalTok{(stay))}
\end{Highlighting}
\end{Shaded}

\begin{verbatim}
stay
    0     1 
0.509 0.491 
\end{verbatim}

\hypertarget{exercises}{%
\subsubsection{Exercises}\label{exercises}}

\begin{enumerate}
\def\labelenumi{\arabic{enumi}.}
\tightlist
\item
  Create a script and call it assignment01. Save your script.
\item
  Download this \href{https://www.rstudio.com/wp-content/uploads/2016/06/r-cheat-sheet.pdf}{cheat-sheet} and go over it. You won't understand most of it right a away. But it will become a useful resource. Look at it often.
\item
  Calculate the square root of 1369 using the \texttt{sqrt()} function.
\item
  Square the number 13 using the \texttt{\^{}} operator.
\item
  What is the result of summing all numbers from 1 to 100?
\end{enumerate}

We take a sample of yearly income in Berlin. The values that we got are: 19395, 22698, 40587, 25705, 26292, 42150, 29609, 12349, 18131, 20543, 37240, 28598, 29007, 26106, 19441, 42869, 29978, 5333, 32013, 20272, 14321, 22820, 14739, 17711, 18749.

\begin{enumerate}
\def\labelenumi{\arabic{enumi}.}
\setcounter{enumi}{5}
\tightlist
\item
  Create the variable \texttt{income} with the values form our Berlin sample in R.
\item
  Describe Berlin income using the appropriate measures of central tendency and dispersion.
\item
  Compute the average deviation without using the \texttt{sd()} function.
\end{enumerate}

Take a look at the Sunday Question (who would you vote for if the general election were next Sunday?) by following this link \href{https://www.wahlrecht.de/umfragen/}{Sunday Question Germany}. You should be able to translate the website into English by right clicking in your browser and clicking ``Translate to English.''

\begin{enumerate}
\def\labelenumi{\arabic{enumi}.}
\setcounter{enumi}{8}
\tightlist
\item
  What is the level of measurement of the variable in the Sunday Question?
\item
  Take the most recent poll and describe what you see in terms of central tendency and dispersion.
\item
  Save your script, which should now include the answers to all the exercises.
\item
  Source your script, i.e.~run the entire script without error message. Clean your script if you get error messages.
\end{enumerate}

\hypertarget{research-design-counterfactuals-forming-hypotheses}{%
\section{Research Design, Counterfactuals, Forming Hypotheses}\label{research-design-counterfactuals-forming-hypotheses}}

\hypertarget{seminar}{%
\subsection{Seminar}\label{seminar}}

In today's seminar, we work with data frames (datasets). We will create our own dataset, we subset datasets (access elements, rows and variables). We load our first dataset into R. We also visualise data using the \texttt{plot()} function. Finally, we estimate a treatment effect in R---our first inference.


\end{document}
